\documentclass[a4paper,11pt]{article}

\usepackage[utf8]{inputenc}
\usepackage[czech]{babel}
\usepackage[left=2cm,top=3cm,text={17cm,24cm}]{geometry}
\usepackage[dvips]{graphicx}
\graphicspath{ {images/} }
\usepackage{listings}
\usepackage{float}
\floatstyle{boxed}
\restylefloat{figure}
\usepackage{url}


\begin{document}

\begin{center}
\textsc{\Huge Fakulta informačních technologií\\
Vysoké učení technické v~Brně\\}

\LARGE \title{Systémy a signály (ISS)\\}
\textbf{ISS Projekt 2017 / 18\\}

\hfill \author{Drahník Lukáš - xdrahn00 - xdrahn00@stud.fit.vutbr.cz}
\end{center}

{\let\newpage\relax\maketitle}

\newpage

\section*{Obsah}
\begin{itemize}
  \item Úvod
  \item Otázka č.1
  \item Otázka č.2
  \item Otázka č.3
  \item Otázka č.4
  \item Otázka č.5
  \item Otázka č.6
  \item Otázka č.7
  \item Otázka č.8
  \item Otázka č.9
  \item Otázka č.10
  \item Otázka č.11
  \item Otázka č.12
  \item Otázka č.13
\end{itemize}

\newpage

\section{Úvod}
Projekt je vypracovaný za pomoci programu GNU Octave [1], verze 4.0.0, program je distribuován zdarma a programu Matlab, verze R2015B, který je dostupný online na fakultním serveru merlin.

\newpage

\section{Otázka č.1}

Délka ve vzorcích: 16000 vzorků\\
Vzorkovací frekvence: 16000 Hz\\
Délka v sekundách: 1 sekunda\\
\newline
Použitá funkce: \textbf{wavread}[2]\
\newline
\newline
Použitý kód:
\lstset{language=Octave}
\begin{lstlisting}[frame=single,breaklines]
  [y, Fs] = wavread('xdrahn00.wav'));
  size(y)
  # 16000
  Fs
  # Fs = 16000
\end{lstlisting}

\section{Otázka č.2}

Použitá funkce: \textbf{fft}
\newline
\newline
Použitý kód:
\lstset{language=Matlab}
\begin{lstlisting}[frame=single,breaklines]
  [y,Fs] = audioread('xdrahn00.wav');
  s = fft(y);
  shalf = s(1:length(s)/2);
  f =(0:Fs/2-1);
  plot(f, abs(shalf));
\end{lstlisting}

\begin{figure}[h]
  \centering
  \includegraphics[scale=0.6]{question-2}
\end{figure}

\section{Otázka č.3}

Maximum modulu spektra: 1.562466826653851e+03 (index 250)\\
\newline
Použitá funkce: \textbf{max}\
\newline
\newline
Použitý kód:
\lstset{language=Matlab}
\begin{lstlisting}[frame=single,breaklines]
  [M, I] = max(abs(shalf));
  # M = 1.562466826653851e+03
  # I = 250
\end{lstlisting}

\section{Otázka č.4}

Aby byl filtr stabilní, musí být všechny póly uvnitř
jednotkové kružnice.\\
\newline
Stabilita filtru: stabilní\\
\newline
Použitá funkce: \textbf{ukazmito}\
\newline
\newline
Použitý kód:
\lstset{language=Matlab}
\begin{lstlisting}[frame=single,breaklines]
  [s, Fs] = audioread ('xdrahn00.wav');
  A = [1 0.2289 0.4662];
  B = [0.2324 -0.4112 0.2324];
  ukazmito(B, A, Fs)
# stabilni...
\end{lstlisting}

\begin{figure}[h]
  \centering
  \includegraphics[scale=0.6]{question-4}
\end{figure}

\section{Otázka č.5}
Horní propusť je frekvenční lineární filtr, který nepropouští signál o nízkých frekvencích.\\
\newline
Typ filtru: horní propust\\
\newline

Obrázek viz. Otázka č.4

\section{Otázka č.6}

Použitá funkce: \textbf{filter}
\newline
\newline
Použitý kód:
\lstset{language=Matlab}
\begin{lstlisting}[frame=single,breaklines]
  [s, Fs] = audioread ('xdrahn00.wav');
  A = [1 0.2289 0.4662];
  B = [0.2324 -0.4112 0.2324];
  y = filter(B, A, s);
  s = fft(y);
  shalf = s(1:length(s)/2);
  f =(0:Fs/2-1);
  plot(f, abs(shalf));
\end{lstlisting}

\begin{figure}[h]
  \centering
  \includegraphics[scale=0.6]{question-6}
\end{figure}

\section{Otázka č.7}

Maximum modulu spektra filtrovaného signálu: 6.624258689332181e+02 (index 5868)\\
\newline
Použitá funkce: \textbf{max}\
\newline
\newline
Použitý kód:
\lstset{language=Matlab}
\begin{lstlisting}[frame=single,breaklines]
  [s, Fs] = audioread ('xdrahn00.wav');
  A = [1 0.2289 0.4662];
  B = [0.2324 -0.4112 0.2324];
  y = filter(B, A, s);
  s = fft(y);
  shalf = s(1:length(s)/2);
  f =(0:Fs/2-1);
  plot(f, abs(shalf));

  [M, I] = max(abs(shalf));
  # M = 6.624258689332181e+02
  # I = 5868
\end{lstlisting}

\section{Otázka č.8}

Použitá funkce: \textbf{xcorr}\\
\newline
Ve vzorcích: 3759
\newline
V sekundách: 0.2349
\newline

\lstset{language=Matlab}
\begin{lstlisting}[frame=single,breaklines]
[s, Fs] = audioread('xdrahn00.wav', [1 16000]);

sequence = [0.5 0.5 -0.5 -0.5];
data = repmat(sequence, 1, 80);

[result, lags] = xcorr(data, s);

[~,I] = max(result);
lagDiff = lag(I)
timeDiff = lagDiff/Fs

lagDiff =
   -3759

timeDiff =
   -0.2349

\end{lstlisting}

\section{Otázka č.9}

\begin{figure}[H]
  \centering
  \includegraphics[scale=0.6]{question-9}
\end{figure}

Použitá funkce: \textbf{xcorr}
\newline
\newline
Použitý kód:
\lstset{language=Matlab}
\begin{lstlisting}[frame=single,breaklines]
  [s, Fs] = audioread('xdrahn00.wav', [1 16000]);

  result = xcorr(s, 'biased');

  N = 51;
  k = (-Fs+1):(Fs-1);
  start_index = find(k==-50);
  end_index = find(k==50);
  plot(k, result(start_index:end_index));

  position = find(k==10);
  value = result(position);
\end{lstlisting}

\section{Otázka č.10}

Hodnota koeficientu R10: 0.0129 (viz. kód Otázka č. 9)\\

\lstset{language=Matlab}
\begin{lstlisting}[frame=single,breaklines]
  position = find(k==10);
  value = result(position);
\end{lstlisting}

\section{Otázka č.11}

\lstset{language=Matlab}
\begin{lstlisting}[frame=single,breaklines]
  [s, Fs] = audioread('xdrahn00.wav', [1 16000]);

  x = linspace(min(min(s)),max(max(s)), 50);
  s1 = cat(1,s,zeros(10,1));
  s2 = cat(1,zeros(10,1),s);
  % nebo s1 = s; a s2 = circshift(s, 10);
  [h, p, r] = hist2opt(s1, s2, x);
  % r = 0.0129

  imagesc(x,x,p); axis xy; c = jet; c = flipud(c); colormap(c); colorbar;
\end{lstlisting}

\begin{figure}[H]
  \centering
  \includegraphics[scale=0.6]{question-11}
\end{figure}

\section{Otázka č.12}

viz. otázka č. 11

\lstset{language=Matlab}
\begin{lstlisting}[frame=single,breaklines]
  hist2: check -- 2d integral should be 1 and is 1
\end{lstlisting}

\section{Otázka č.13}

Hodnota koeficientu R10: 0.0129 (viz. kód Otázka č. 11)\\

\nocite{*}

%% BIBLIOGRAPHY
\bibliography{local}
\bibliographystyle{plain}

\newpage
\thispagestyle{empty}

%% IMAGES
%% \listoffigures

\end{document}
%% END OF FILE
