\documentclass[a4paper,11pt]{article}

\usepackage[utf8]{inputenc}
\usepackage[czech]{babel}
\usepackage[left=2cm,top=3cm,text={17cm,24cm}]{geometry}
\usepackage[dvips]{graphicx}
\graphicspath{ {images/} }
\usepackage{listings}
\usepackage{url}


\begin{document}

\begin{center}
\textsc{\Huge Fakulta informačních technologií\\
Vysoké učení technické v~Brně\\}

\LARGE \title{Systémy a signály (ISS)\\}
\textbf{ISS Projekt 2017 / 18\\}

\hfill \author{Drahník Lukáš - xdrahn00 - xdrahn00@stud.fit.vutbr.cz}
\end{center}

{\let\newpage\relax\maketitle}

\newpage

\section*{Obsah}
\begin{itemize}
  \item Úvod
  \item Otázka č.1
  \item Otázka č.2
  \item Otázka č.3
  \item Otázka č.4
  \item Otázka č.5
  \item Otázka č.6
  \item Otázka č.7
  \item Otázka č.8
  \item Otázka č.9
  \item Otázka č.10
  \item Otázka č.11
  \item Otázka č.12
  \item Otázka č.13
\end{itemize}

\newpage

\section{Úvod}
Projekt je vypracovaný za pomoci programu GNU Octave [1], verze 4.0.0, program je distribuován zdarma a programu Matlab, verze R2015B, který je dostupný online na serveru merlin.fit.vutbr.cz.

\newpage

\section{Otázka č.1}

Délka ve vzorcích: 16000 vzorků\\
Vzorkovací frekvence: 1 Hz\\
Délka v sekundách: 1 sekunda\\
\newline
Použitá funkce: \textbf{wavread}[2]\
\newline
\newline
Použitý kód:
\lstset{language=Octave}
\begin{lstlisting}[frame=single,breaklines]
  [sig, fs] = wavread('xdrahn00.wav', "size")
  # sig =  16000
  # fs =  1
  length(sig)/fs
  # ans =  1
\end{lstlisting}

\section{Otázka č.2}

Použitá funkce: \textbf{vyber}
\newline
\newline
Použitý kód:
\lstset{language=Octave}
\begin{lstlisting}[frame=single,breaklines]
  [s, Fs] = audioread ('xdrahn00.wav');
  x = vyber(s, 0.5, 0.5, Fs);
\end{lstlisting}

\begin{figure}[h]
  \centering
  \includegraphics[scale=0.6]{question-2}
\end{figure}

\section{Otázka č.3}

Maximum modulu spektra: 126Hz (19.3915)\\
\newline
Použitá funkce: \textbf{vyber2}\
\newline
\newline
Použitý kód:
\lstset{language=Octave}
\begin{lstlisting}[frame=single,breaklines]
  [U, I] = max(G)
  # I * Fs = 126
\end{lstlisting}

\section{Otázka č.4}

Stabilita filtru: Nestabilní\\
\newline
Použitá funkce: \textbf{ukazmito}\
\newline
\newline
Použitý kód:
\lstset{language=Octave}
\begin{lstlisting}[frame=single,breaklines]
  [s, Fs] = audioread ('xdrahn00.wav');
  A = [0.2289 0.4662];
  B = [0.2324 -0.4112 0.2324];
  ukazmito(A, B, Fs)
\end{lstlisting}

\begin{figure}[h]
  \centering
  \includegraphics[scale=0.6]{question-4}
\end{figure}

\nocite{*}

%% BIBLIOGRAPHY
\bibliography{local}
\bibliographystyle{plain}

\newpage
\thispagestyle{empty}

%% IMAGES
%% \listoffigures

\end{document}
%% END OF FILE
